\documentclass[twoside]{article}
\usepackage{amsmath,amssymb,graphicx}
\usepackage{hyperref}
\pagestyle{empty}

%------------------------------------------------------------
\begin{document}

%%%%%%%%% TITLE of abstract %%%%%%%%%

\begin{center}
	\textbf{\large An introduction to Julia as a tool in scientific computing}
\end{center}

%%%%%%%%%  SPEAKER %%%%%%%%%

\begin{center}
	Luiz M. Faria\\
	\emph{INRIA, Laboratoire POEMS}\\
	\texttt{luiz.maltez-faria@inria.fr}
\end{center}

\bigskip

%%%%%%%%%  text of ABSTRACT to be included below  %%%%%%
\noindent

In this presentation, we will take a look at the Julia programming language as a tool for
scientific computing. Julia's central promise is to combine performance comparable to that
of lower-level languages like C or Fortran with the usability of higher-level languages like
Python or Matlab. The goal of this talk is to introduce the language and examine this claim
empirically, through a simple but representative example.

After taking a shallow dive into the main language features, we will focus on the concrete
problem of computing the many-body interactions in a gravitational system (the $N$-body
problem). The naive algorithm is essentially a double loop over the particles, and its large
arithmetic intensity ($\mathcal{O}(N^2)$ flops with $\mathcal{O}(N)$ bytes) makes it a good
candidate for performance exploration and optimization. We will explore the implementation
in Julia, C, and Python, and compare their performance and readability. We will then move on
to optimizing the Julia code by using (i) SIMD instructions and (ii)
multi-threading. Finally, we will showcase how Julia's high-level abstractions --- in
particular parametric types and multiple dispatch --- can be used to make the code more
generic and reusable without sacrificing performance. At the end, I will share some thoughts
on where I think Julia shines, and where it falls short compared to more mature languages
like C++.

\paragraph{Note:} If you want to try some of the examples live, feel free to
\href{https://julialang.org/downloads/}{install Julia} before the talk! And for an IDE-like
experience, you can also install e.g. \href{https://www.julia-vscode.org/}{the Julia extension for
	VS Code}.


%%%%%%%%%  end text of ABSTRACT   %%%%%%%%%%

\bigskip

% %%%%%%%%%  REFERENCES %%%%%%%%

\smallskip

%PAPER
%\bibliographystyle{plain}
%\bibliography{references}

% \noindent
% \textbf{References}
% \begin{enumerate}
%  \item[\rm{[1]}]
%  L. M. Faria, C. Pérez-Arancibia, M. Bonnet. ``General-purpose kernel
%  regularization of boundary integral equations via density interpolation".
%  \textit{Computer methods in applied mechanics and engineering}, 2021.
% \end{enumerate}

% % %%%%%% CO-AUTHORS %%%%%%%
% \hrule
% \medskip


\end{document}

%%% Local Variables:
%%% mode: latex
%%% TeX-master: t
%%% End:
